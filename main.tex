\documentclass[12pt,a4paper]{article}
\pagestyle{plain}

% ---------- 基本レイアウト ----------
\usepackage[margin=25mm]{geometry}
\usepackage{setspace}
\onehalfspacing

% ---------- フォント・文字コード ----------
\usepackage[T1]{fontenc}
\usepackage[utf8]{inputenc}
\usepackage{lmodern} % 必要なら newtxtext,newtxmath に変更可

% ---------- 数式・図表 ----------
\usepackage{amsmath,amssymb}
\usepackage{graphicx}
\usepackage{booktabs}
\usepackage{siunitx}
\usepackage{threeparttable}
\usepackage{caption}
\captionsetup{labelfont=bf}

% ---------- ハイパーリンク ----------
\usepackage[hidelinks]{hyperref}

% ---------- 参考文献(APSA風) ----------
\usepackage[authoryear]{natbib}
\bibliographystyle{apsr}

% ---------- タイトル情報 ----------
\title{Beneath the Surface: Measuring Hidden Indifference\\
in Japanese Territorial Attitudes}

\author{Takeru Eto\thanks{JX Press Corporation / Graduate School of Arts and Sciences, The University of Tokyo. Email:
\texttt{takeru0219@gmail.com}.}}

\date{Draft: \today}

\begin{document}

\maketitle

\begin{abstract}
Japan is currently engaged in three major territorial disputes: the Northern Territories
(known as the Kuril Islands to Russia), Takeshima (Liancourt Rocks, or Dokdo to South
Korea), and the Senkaku Islands (Diaoyu to China and Tiaoyutai to Taiwan).
Official government surveys consistently report high levels of public concern regarding these
territories. However, public opinion on matters of national identity and territory is not
always straightforward, and respondents may overstate their concern due to social
desirability bias. Expressing indifference on a matter of national sovereignty could be
perceived as unpatriotic, leading respondents to conceal their genuine lack of a strong
opinion.

To address this issue, this paper employs a list experiment embedded in a large-scale online
survey of 4{,}500 Japanese citizens. The design estimates the proportion of the population
that is privately indifferent to each territorial dispute. The results reveal substantial levels of
hidden indifference: approximately 20\% of respondents do not have a strong opinion on the
sovereignty of Takeshima, and around 12\% are indifferent to the Senkaku Islands dispute.
For the Northern Territories, the estimate is smaller and statistically indistinguishable from
zero. These findings challenge the conventional view of a uniformly concerned public and
suggest that the domestic political constraints facing Japanese leaders may be more flexible
than commonly assumed.
\end{abstract}

\bigskip
\noindent\textbf{Keywords:} territorial disputes; public opinion; social desirability bias; list experiment; Japan

\section{Introduction}

Japan is currently engaged in three major territorial disputes: the Northern Territories (known
as the Kuril Islands to Russia), Takeshima (Liancourt Rocks, or Dokdo to South Korea), and
the Senkaku Islands (Diaoyu to China and Tiaoyutai to Taiwan) \citep{bukh2018}. Official
government surveys consistently report high levels of public concern regarding these
territories. For instance, a 2023 survey by the Cabinet Office found that 63.6\% of respondents
were ``concerned'' about the Takeshima issue \citep{cabinet2023takeshima}, while a 2024
survey reported that 78.4\% were concerned about the Senkaku Islands
\citep{cabinet2024senkaku}. These figures are often cited as evidence of a strong, unified
national consensus demanding a firm governmental stance.

However, public opinion on matters of national identity and territory is not always
straightforward. Governments may leverage territorial issues to stoke nationalist sentiment
for domestic political gain \citep{fang2022}, as has been argued in the case of Japan's policy
on the Northern Territories \citep{bukh2018}. Conversely, strong public sentiment can
constrain a government's diplomatic flexibility, narrowing the ``zone of possible agreement''
by making any perceived concession politically costly \citep{loizides2022}. Given these
dynamics, a critical question arises: Are the Japanese people truly as concerned with these
territorial disputes as direct surveys suggest?

This paper posits that conventional surveys may overestimate public engagement due
to social desirability bias (SDB), the tendency of respondents to give answers they believe are
socially acceptable rather than expressing their true beliefs. Expressing indifference on a
matter of national sovereignty could be perceived as unpatriotic, leading respondents to
conceal their genuine lack of a strong opinion. Moreover, as these surveys are conducted by
the Japanese government, they inquire whether respondents are concerned about each issue
and what actions they believe the Japanese government should undertake, without eliciting
respondents' claims or opinions regarding each disputed territory.

To overcome this methodological challenge, this study employs a list experiment, an
indirect questioning technique designed to elicit more truthful answers on sensitive topics
\citep{blair2012}. By embedding this experiment within a large-scale online survey, this
research provides a quantitative estimate of the proportion of the Japanese public that is
privately indifferent to these territorial disputes. The findings suggest that a considerable
portion of the population does not hold a strong opinion, revealing a more nuanced and
complex reality of public attitudes than is commonly assumed.

\section{Literature Review: Nationalism, Public Opinion, and Sensitive Attitudes}

\subsection{Public Opinion and Foreign Policy in Territorial Disputes}

The link between public opinion and foreign policy is a central theme in international
relations. In democratic states, leaders must remain attentive to public sentiment, as it can
impose significant political costs on those who are seen as compromising on core national
interests. Territorial disputes are particularly potent in this regard, as land is often framed as
an “indivisible’’ part of the national identity. Research shows that public opinion can drive
governments toward harder stances and limit their ability to negotiate peaceful resolutions
\citep{fang2022,loizides2022}.

However, the causal relationship is not always unidirectional. Governments also
actively shape public opinion. \citet{bukh2018} argues that the Japanese government
historically leveraged the Northern Territories issue as a tool to foster anti-Soviet sentiment,
a strategy that proved successful in domestic politics but ultimately hindered diplomatic
progress. This highlights the complex interplay between elite-led narratives and grassroots
public sentiment. While official discourse in Japan portrays a nation united in its concern for
its territories, no academic research has systematically investigated the “true’’ underlying
attitudes of the public, free from the pressures of social desirability.

Prior research has shown that public attitudes toward territorial disputes are far from uniform.
In particular, \citet{tanaka_2015} demonstrates that proximity to disputed areas and
economic considerations systematically shape citizens’ degree of hawkishness. His findings
highlight that even in highly salient territorial conflicts, sub-national heterogeneity exists,
challenging the assumption of a monolithic nationalist public. These insights imply that
variation may exist not only in how strongly people feel about a dispute but also in whether
they hold a strong opinion at all—an issue that direct surveys are ill-equipped to detect.

Existing work also highlights the emotional mechanisms through which
territorial threats shape political attitudes \citep{kobayashi_katagiri_2018}. Threats from rival
states can evoke anger and heighten nationalist pressure, thereby making expressions of
indifference socially costly. The current findings therefore suggest a parallel mechanism:
external threats may not only mobilize anger but also create social pressure to express
concern, prompting indifferent citizens to conceal their true attitudes.

\subsection{Measuring Sensitive Attitudes: The List Experiment}

Directly asking about sensitive topics—such as racial prejudice, corruption, or support for
extremist ideologies—is notoriously difficult due to social desirability bias
\citep{blair2012}. Respondents may mask their true views to avoid social sanction. This
phenomenon has been observed in various political contexts, from citizens concealing dissent
against the war in Ukraine to Taiwanese citizens hiding pro-unification attitudes
\citep{kramon2019,chapkovski2022,wu2024}.

The list experiment, also known as the Item Count Technique (ICT), offers a solution.
This indirect questioning method presents two randomly assigned groups with a list of items.
The control group receives non-sensitive “control’’ items, while the treatment group receives
the same list plus one sensitive item. Respondents report only the total number of applicable
items, not which ones. By providing this veil of privacy, the technique encourages more
honest responses and enables researchers to detect attitudes that remain hidden in direct
surveys \citep{blair2012,glynn2013,kramon2019}.

Consistent with this, \citet{kim_2015} finds weak demographic predictors of territorial
attitudes using direct surveys. The present study suggests one reason for such weak
associations: a large portion of respondents may be giving socially desirable answers,
masking true variation in latent attitudes. Indirect methods thus uncover structure that direct
questioning fails to detect, providing a more accurate picture of the distribution of public
opinion on territorial issues.

While recent work has begun to employ list experiments to
study attitudes toward territorial concessions in Ukraine and to reassess survey estimates
in the Japan–China dispute, no study to date has systematically examined hidden indifference toward multiple territorial disputes
in a non-war setting such as contemporary Japan.

In the context of Russia’s invasion of Ukraine, \citet{daniels2025} uses a list experiment
to assess Ukrainians’ willingness to concede territory—specifically, Crimea.
The study finds that the estimated share supporting concessions is very small (around 5–6 percent)
and closely matches direct-question estimates, suggesting limited social desirability bias in that particular wartime context.

Recent studies on Ukraine similarly employ list experiments to gauge support
for territorial concessions during wartime, finding that only a small minority of
citizens are willing to “trade land for peace” and that opposition to concessions is largely
genuine rather than a product of social desirability bias \citep{popeleches_robertson_2025}.

\section{Research Design and Methodology}

This study addresses the research question: ``How many Japanese citizens are genuinely
indifferent to the country's territorial disputes?'' An online, opt-in panel survey was
conducted on June 28--29, 2025, with a sample of 4{,}500 Japanese citizens aged 18 to 99.
Quotas were set for gender and region (47 prefectures) to enhance the representativeness of
the sample.

Respondents were randomly assigned to either a control group or one of three
treatment groups, each focused on a different territory (Takeshima, the Senkaku Islands, or
the Northern Territories).

\paragraph{Control Group.} Respondents were shown a list of four non-sensitive items and
asked how many applied to them. The items were designed to be politically neutral and
varied in their likely applicability to the general population:
\begin{enumerate}
  \item I would rather watch movies in a theater than at home.
  \item I tend to plan my trips in detail beforehand.
  \item I have made a donation(s) through the Furusato Nozei (Hometown Tax Donation) system before.
  \item I have purchased a physical book or magazine from a bookstore within the last month.
\end{enumerate}

\paragraph{Treatment Groups.} Respondents received the same four non-sensitive items plus
one sensitive item. The sensitive item was phrased to capture indifference:
\begin{quote}
  Personally, I do not have a strong opinion on the sovereignty of [Takeshima / the
  Senkaku Islands / the Northern Territories].
\end{quote}

The difference in the mean count of items between a treatment group and the control
group provides the estimated proportion of respondents who agree with the sensitive
item---that is, the proportion who are indifferent to that specific territorial dispute.

\section{Results}

The results of the list experiment reveal a substantial level of hidden indifference toward
Japan's territorial disputes, challenging the narrative of a uniformly concerned public.

\subsection{Overall Estimated Indifference}

As shown in Figure~\ref{fig:territory}, the estimated proportion of respondents who are
indifferent varies by territory:
\begin{itemize}
  \item \textbf{Takeshima}: An estimated 19.8\% of the Japanese public does not have a strong
  opinion on the sovereignty of Takeshima. This result is statistically significant, with a
  95\% confidence interval that does not include zero.
  \item \textbf{Senkaku Islands}: An estimated 12.0\% are indifferent toward the Senkaku Islands
  dispute. This finding is also statistically significant.
  \item \textbf{Northern Territories}: The estimated proportion of indifference is 5.7\%.
  However, the confidence interval for this estimate includes zero, meaning the result is
  not statistically significant and cannot be distinguished from a null effect.
\end{itemize}

\begin{figure}[t]
  \centering
  % 自分のファイル名に合わせて差し替えてください
  % 図のファイルが存在しない場合は、以下の行をコメントアウトしてください
  \IfFileExists{figure_le_indifference_by_territory.pdf}{%
    \includegraphics[width=0.8\textwidth]{figure_le_indifference_by_territory.pdf}%
  }{%
    \fbox{\parbox{0.8\textwidth}{\centering\vspace{2cm}[Figure 1: Estimated Proportion of Indifference by Territory]\\\vspace{2cm}}}%
  }
  \caption{Estimated Proportion of Indifference by Territory}
  \label{fig:territory}
\end{figure}

These findings are striking. For both Takeshima and the Senkaku Islands, a sizable
minority---one in five and one in eight citizens, respectively---privately harbors an
indifference that is not captured by conventional polling methods. The lack of a statistically
significant result for the Northern Territories may suggest that public opinion on this issue is
either genuinely more consolidated or that the list experiment was less effective in this
specific context for reasons that warrant further investigation.

\subsection{Analysis by Subgroups}

Preliminary analysis of demographic subgroups reveals further nuances, which are shown in
Figures~\ref{fig:region} and \ref{fig:age}. Figure~\ref{fig:region} shows the results for
regions, and Figure~\ref{fig:age} shows the results for age groups.

For the Takeshima dispute, indifference appears to be higher among men than women,
and among younger (30--49) and older (70 and above) age groups compared to middle-aged
(50--69) respondents. A regional breakdown shows significant variation, with notably high
levels of indifference in Tokyo, while the estimate for the Chugoku region (geographically
closest to the dispute) is statistically insignificant and negative, suggesting potential
resistance to expressing indifference even indirectly.

Similar patterns emerge for the Senkaku Islands, with men and both younger and
older cohorts showing higher levels of indifference. As with Takeshima, the estimate for
indifference in Tokyo is high, while in Kyushu (geographically closer), the estimate is
statistically insignificant. These regional patterns suggest that geographic proximity to a
disputed territory may correlate with stronger, more publicly aligned opinions. However,
more rigorous analysis with demographic covariates is needed to confirm these initial
observations.

\begin{figure}[t]
  \centering
  % 自分のファイル名に合わせて差し替えてください
  % 図のファイルが存在しない場合は、以下の行をコメントアウトしてください
  \IfFileExists{figure_le_indifference_by_PRblock_facet.pdf}{%
    \includegraphics[width=1.0\textwidth]{figure_le_indifference_by_PRblock_facet.pdf}%
  }{%
    \fbox{\parbox{1.0\textwidth}{\centering\vspace{2cm}[Figure 2: Estimated Proportion of Indifference by Territory and Region]\\\vspace{2cm}}}%
  }
  \caption{Estimated Proportion of Indifference by Territory and Region\\
  \emph{Note:} Prefectures are grouped according to the PR block in lower house elections.}
  \label{fig:region}
\end{figure}

\begin{figure}[t]
  \centering
  % 自分のファイル名に合わせて差し替えてください
  % 図のファイルが存在しない場合は、以下の行をコメントアウトしてください
  \IfFileExists{figure_le_indifference_by_age.pdf}{%
    \includegraphics[width=1.0\textwidth]{figure_le_indifference_by_age.pdf}%
  }{%
    \fbox{\parbox{1.0\textwidth}{\centering\vspace{2cm}[Figure 3: Estimated Proportion of Indifference by Territory and Age Group]\\\vspace{2cm}}}%
  }
  \caption{Estimated Proportion of Indifference by Territory and Age Group}
  \label{fig:age}
\end{figure}

\subsection{ICT Regression Analysis}

To further examine the demographic and regional patterns of indifference, we conducted
regression analyses using the Item Count Technique (ICT) framework. Table~\ref{tab:ict_regression}
presents the results of ordinary least squares (OLS) models for each of the three territorial
disputes. The dependent variable is the count of items respondents indicated applied to them.
The key coefficients of interest are the Treatment indicator and its interactions with
demographic and regional variables, which capture latent indifference toward each dispute.

\begin{table}[htbp]
\centering
\caption{ICT Regression Results for Three Territorial Disputes (Key Coefficients)}
\label{tab:ict_regression}
\begin{threeparttable}
\scriptsize % small → scriptsize に変更
\setlength{\tabcolsep}{2pt} % 列の左右の余白を狭くする
\renewcommand{\arraystretch}{1.1} % 行間を詰める
\begin{tabular}{lccc}
\toprule
& Takeshima & Senkaku & Northern Territories \\
\midrule
Treatment (List) & \num{-0.120} & \num{0.235} & \num{0.348} \\
& (\num{0.296}) & (\num{0.316}) & (\num{0.305}) \\
Treatment × Male & \num{-0.157} & \num{-0.175}+ & \num{-0.152} \\
& (\num{0.096}) & (\num{0.093}) & (\num{0.093}) \\
Treatment × Other gender & \num{-0.565} & \num{-0.309} & \num{0.005} \\
& (\num{0.387}) & (\num{0.397}) & (\num{0.396}) \\
Treatment × Age: 30s & \num{-0.010} & \num{0.057} & \num{-0.634}* \\
& (\num{0.263}) & (\num{0.261}) & (\num{0.265}) \\
Treatment × Age: 40s & \num{0.165} & \num{0.092} & \num{-0.528}* \\
& (\num{0.252}) & (\num{0.254}) & (\num{0.257}) \\
Treatment × Age: 50s & \num{-0.152} & \num{-0.205} & \num{-0.656}* \\
& (\num{0.251}) & (\num{0.249}) & (\num{0.256}) \\
Treatment × Age: 60s & \num{-0.075} & \num{0.097} & \num{-0.323} \\
& (\num{0.256}) & (\num{0.254}) & (\num{0.260}) \\
Treatment × Age: 70+ & \num{-0.074} & \num{-0.124} & \num{-0.408} \\
& (\num{0.249}) & (\num{0.248}) & (\num{0.255}) \\
Treatment × Hokkaido & \num{0.498}+ & \num{0.146} & \num{0.545}+ \\
& (\num{0.286}) & (\num{0.292}) & (\num{0.285}) \\
Treatment × Tohoku & \num{0.435} & \num{0.011} & \num{0.159} \\
& (\num{0.278}) & (\num{0.299}) & (\num{0.280}) \\
Treatment × Kita-Kanto & \num{0.339} & \num{-0.073} & \num{0.202} \\
& (\num{0.244}) & (\num{0.260}) & (\num{0.239}) \\
Treatment × Minami-Kanto & \num{0.292} & \num{-0.018} & \num{0.251} \\
& (\num{0.232}) & (\num{0.250}) & (\num{0.227}) \\
Treatment × Tokyo & \num{0.710}** & \num{0.133} & \num{0.356} \\
& (\num{0.238}) & (\num{0.258}) & (\num{0.234}) \\
Treatment × Hokuriku-Shinetsu & \num{0.558}* & \num{0.263} & \num{0.449} \\
& (\num{0.279}) & (\num{0.299}) & (\num{0.275}) \\
Treatment × Tokai & \num{0.263} & \num{-0.049} & \num{0.101} \\
& (\num{0.241}) & (\num{0.258}) & (\num{0.235}) \\
Treatment × Kinki & \num{0.579}* & \num{-0.044} & \num{0.383}+ \\
& (\num{0.227}) & (\num{0.245}) & (\num{0.219}) \\
Treatment × Shikoku & \num{0.139} & \num{-0.339} & \num{-0.055} \\
& (\num{0.356}) & (\num{0.383}) & (\num{0.339}) \\
Treatment × Kyushu & \num{0.556}* & \num{-0.054} & \num{0.285} \\
& (\num{0.255}) & (\num{0.275}) & (\num{0.253}) \\
\midrule
Num.Obs. & \num{2253} & \num{2275} & \num{2266} \\
R2       & \num{0.038} & \num{0.026} & \num{0.033} \\
F        & \num{2.469} & \num{1.690} & \num{2.154} \\
RMSE     & \num{1.10}  & \num{1.09}  & \num{1.08} \\
\bottomrule
\end{tabular}
\begin{tablenotes}[flushleft]
\footnotesize
\item + p $<$ 0.1, * p $<$ 0.05, ** p $<$ 0.01, *** p $<$ 0.001
\item OLS models of list experiment item counts. Coefficients on Treatment and its interactions capture latent indifference; coefficients without Treatment are omitted for brevity.
\end{tablenotes}
\end{threeparttable}
\end{table}


The results in Table~\ref{tab:ict_regression} reveal several noteworthy patterns. For Takeshima,
the interaction terms show significantly higher indifference in Tokyo, Hokuriku-Shinetsu, Kinki,
and Kyushu regions. For the Senkaku Islands, the treatment effect shows a marginally significant
negative interaction with male respondents. For the Northern Territories, younger age groups
(30s, 40s, and 50s) show significantly lower treatment effects, suggesting less indifference
among these cohorts compared to the youngest reference group.


\section{Discussion and Conclusion}

This study set out to measure the ``hidden indifference'' of the Japanese public toward
territorial disputes. The results from the list experiment strongly suggest that public opinion is
not the monolith that direct surveys often portray. The finding that approximately 20\% of the
public is indifferent to the Takeshima issue and 12\% to the Senkaku Islands issue has
significant implications for understanding Japanese politics and foreign policy.

First, it suggests that a silent minority may be more open to diplomatic compromise
than is commonly believed. While vocal nationalist groups and official rhetoric create an
impression of uniform public resolve, there exists a latent ``zone of possible agreement'' that
political leaders could potentially activate if they chose to lead public opinion rather than
simply follow it. The government's fear of public backlash, which has been cited as a reason
for rejecting diplomatic overtures in the past, may be based on an overestimation of public
rigidity.

Second, the findings are a powerful reminder of the importance of methodology in
survey research on sensitive topics. Social desirability bias appears to significantly mask true
attitudes on territorial issues in Japan. Future research and media reporting on this topic
should exercise caution when interpreting the results of direct-question polls and
acknowledge the potential for such biases.

This research is not without limitations. The use of an online opt-in panel introduces
potential selection bias. Furthermore, the phrasing of the sensitive item---``I do not have a
strong opinion''---may be open to some interpretation. Future work should analyze the
demographic and political characteristics of the ``indifferent'' group to understand who they
are, and employ other methods to triangulate these findings.

In conclusion, by looking beneath the surface of official polling data, this study
reveals a more complex and nuanced landscape of Japanese public opinion. A sizable portion
of the citizenry is privately indifferent to territorial sovereignty, a fact hidden by the pressures
to conform to a nationalist consensus. Recognizing this hidden indifference is a critical first
step toward a more accurate understanding of the domestic political constraints and
opportunities facing Japanese leaders as they navigate these contentious diplomatic waters.

\section*{Acknowledgments}

I thank JX Press Corporation for providing access to the survey data used in this study.
The views expressed here are my own and do not represent those of the company.

\section*{Data Availability}
The survey data used in this paper were provided by JX Press Corporation and
cannot be publicly shared due to contractual and confidentiality restrictions.
Replication code using simulated example data is available upon request.

\section*{Conflict of Interest}
The author has an outsourcing contract with JX Press Corporation,
which provided access to the data used in this study.
The company had no role in the design, analysis, or conclusions of this research.


\bibliography{references}

\end{document}
