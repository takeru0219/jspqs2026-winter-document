\documentclass[12pt,a4paper]{article}
\pagestyle{plain}

% ---------- 基本レイアウト ----------
\usepackage[margin=25mm]{geometry}
\usepackage{setspace}
\onehalfspacing

% ---------- フォント・文字コード ----------
\usepackage[T1]{fontenc}
\usepackage[utf8]{inputenc}
\usepackage{lmodern} % 必要なら newtxtext,newtxmath に変更可

% ---------- 日本語サポート ----------
\usepackage{CJKutf8} % 日本語表示用
\newcommand{\jptext}[1]{\begin{CJK}{UTF8}{ipxm}#1\end{CJK}} % IPAex明朝フォント使用

% ---------- 数式・図表 ----------
\usepackage{amsmath,amssymb}
\usepackage{graphicx}
\usepackage{booktabs}
\usepackage{siunitx}
\usepackage{threeparttable}
\usepackage{caption}
\captionsetup{labelfont=bf}

\usepackage{placeins} % \FloatBarrier コマンド用
\usepackage{enumitem}

% ---------- ハイパーリンク ----------
\usepackage[hidelinks]{hyperref}

% ---------- 参考文献(APSA風) ----------
\usepackage[authoryear]{natbib}
\bibliographystyle{apsr}

% ---------- タイトル情報 ----------
\title{Beneath the Surface: Measuring Hidden Indifference\\
in Japanese Territorial Attitudes}

\author{Takeru Eto\thanks{JX Press Corporation / Graduate School of Arts and Sciences, The University of Tokyo. Email: \texttt{takeru0219@gmail.com}.}}

\date{Draft: \today}

\begin{document}

\maketitle

\begin{abstract}
Japan is currently engaged in three major territorial disputes: the Northern Territories, Takeshima, and the Senkaku Islands. While official government surveys consistently report high levels of public concern, respondents may overstate their engagement due to social desirability bias. To address this, this paper employs a list experiment through an online opt-in survey of 4,500 Japanese citizens to estimate the proportion of the population that is privately indifferent to each dispute. The results reveal substantial levels of hidden indifference: approximately 19\% for Takeshima, 11\% for the Senkaku Islands, and 5\% for the Northern Territories. Crucially, this indifference is not randomly distributed but is systematically structured by socioeconomic and geographic factors. The analysis shows that indifference is significantly higher among individuals in precarious employment. Furthermore, residents of metropolitan areas, such as Tokyo and Kinki region, exhibit greater indifference toward Takeshima and the Senkaku Islands, while people in closer proximity to these territories tend to be more engaged. Interestingly, however, residents in Hokkaido, the region closest to the Northern Territories, exhibit higher latent indifference than the national average, potentially indicating a divergence between local pragmatic interests and national narratives. These findings challenge the conventional view of a uniformly concerned public and suggest that domestic political constraints on diplomatic compromise may be more flexible than commonly assumed.

\end{abstract}

\bigskip
\noindent\textbf{Keywords:} territorial disputes; public opinion; social desirability bias; list experiment; Japan

\section{Introduction}

Japan is currently engaged in three major territorial disputes: the Northern Territories (known as the Kuril Islands in Russia), Takeshima (Dokdo in South Korea), and the Senkaku Islands (Diaoyu in China and Tiaoyutai in Taiwan) \citep{bukh2018}. Official government surveys consistently report high levels of public concern regarding these territories. For instance, a 2023 survey by the Cabinet Office found that 63.6\% of respondents were ``concerned'' about the Takeshima issue \citep{cabinet2023takeshima}, while a 2024 survey reported that 78.4\% were concerned about the Senkaku Islands \citep{cabinet2024senkaku}. These figures are often cited as evidence of a strong, unified national consensus demanding a firm governmental stance.

However, public opinion on matters of national identity and territory is not always straightforward. Governments may leverage territorial issues to accelerate nationalist sentiment for domestic political gain \citep{fang2022}, as has been argued in the case of Japan's policy on the Northern Territories \citep{bukh2018}. Conversely, strong public sentiment can constrain a government's diplomatic flexibility, narrowing the ``zone of possible agreement'' by making any perceived concession politically costly \citep{loizides2022}. Given these dynamics, a critical question arises: Are the Japanese people truly as concerned with these territorial disputes as direct surveys suggest?

This paper posits that conventional surveys may overestimate public engagement due to social desirability bias (SDB), the tendency of respondents to give answers they believe are socially acceptable rather than expressing their true beliefs. Expressing indifference on a matter of national sovereignty could be perceived as unpatriotic, leading respondents to conceal their true attitudes.

To overcome this challenge, this study employs a list experiment, an indirect questioning technique designed to elicit more truthful answers on sensitive topics \citep{blair2012, dimewiki_listexperiments}. By embedding this experiment within an online survey, this research provides a quantitative estimate of the proportion of the Japanese public that is privately indifferent to these territorial disputes. The findings suggest that a considerable portion of the population neither holds a strong opinion nor cares about which country has these territories, revealing a more nuanced and complex reality of public attitudes than is commonly expected.

\section{Literature Review: Nationalism, Public Opinion, and Sensitive Attitudes}

\subsection{Public Opinion and Foreign Policy in Territorial Disputes}

The link between public opinion and foreign policy is a central theme in international relations. \citet{putnam1988twolevel} argues that diplomatic negotiations should be understood as ``two-level games,'' which involve interactions between domestic and international factors. In democratic states, leaders must be attentive to public sentiment, as it can impose significant political costs on those who are seen as compromising on core national interests. Territorial disputes are particularly potent in this regard, as land is often framed as an ``indivisible'' part of the national identity. Research shows that public opinion can drive governments toward harder stances and limit their ability to negotiate peaceful resolutions \citep{fang2022,loizides2022}.

However, the causal relationship is not always unidirectional. Governments also actively shape public opinion. \citet{bukh2018} argues that the Japanese government historically leveraged the Northern Territories issue as a tool to nurture anti-Soviet sentiment, a strategy that proved successful in domestic politics but ultimately hindered diplomatic progress. This highlights the complex interaction between elite-led narratives and grassroots public sentiment. While official discourse in Japan highlights a unified national concern on territorial issues, the actual attitudes of citizens may be more heterogeneous.

Prior research has shown that public attitudes toward territorial disputes are far from uniform. In particular, \citet{tanaka_2015} demonstrates that proximity to disputed areas and economic considerations shape citizens’ degree of hawkishness. His findings highlight that even in highly salient territorial conflicts, sub-national heterogeneity exists, challenging the assumption of a monolithic nationalist public. These insights imply that variation may exist not only in how strongly people feel about a dispute but also in whether they hold a strong opinion at all, which is difficult to capture in direct surveys.

Existing work also highlights the emotional mechanisms through which territorial threats shape political attitudes \citep{kobayashi_katagiri_2018}. Threats from rival states can evoke anger and heighten nationalist pressure, thereby making expressions of indifference socially costly. The current findings therefore suggest a parallel mechanism: external threats may not only mobilize anger but also create social pressure to express concern, prompting indifferent citizens to conceal their true attitudes.

\subsection{Measuring Sensitive Attitudes: The List Experiment}

Directly asking about sensitive topics, such as racial prejudice, corruption, or support for extremist ideologies, is notoriously difficult due to social desirability bias \citep{blair2012}. Respondents may hide their true views due to implicit social pressure. This phenomenon has been observed in various political contexts, from citizens concealing dissent against the war in Ukraine to Taiwanese citizens hiding pro-unification attitudes \citep{kramon2019,chapkovski2022,wu2024}.

The list experiment, also known as the Item Count Technique (ICT), offers a solution. This indirect questioning method presents two randomly assigned groups with a list of items.
The control group receives non-sensitive ``control'' items, while the treatment group receives the same list plus one ``sensitive'' item. Respondents report only the total number of applicable items, not which questions they agree with. The technique thus protects individual privacy regarding the sensitive item and enables researchers to estimate the proportion of respondents who agree with the sensitive item by comparing the average counts between the two groups \citep{blair2012,glynn2013,kramon2019}.

Consistent with this, \citet{kim_2015} finds weak demographic predictors of territorial attitudes using direct surveys. The present study suggests one reason for such weak associations: a large portion of respondents may be giving socially desirable answers, masking true variation in latent attitudes. Indirect methods thus reveal structure that direct questioning cannot detect, providing a more accurate picture of the distribution of public opinion on territorial issues.

While recent work has begun to employ list experiments to study attitudes toward territorial concessions in Ukraine, no study to date has systematically examined hidden indifference toward multiple territorial disputes in a non-war setting such as contemporary Japan. In the context of Russia's invasion of Ukraine, \citet{daniels2025} uses a list experiment to assess Ukrainians' willingness to concede territory---specifically, Crimea. The study finds that the estimated share supporting concessions is very small (around 5--6 percent) and closely matches direct-question estimates, suggesting that social desirability bias is limited in that particular wartime context.
\citet{popeleches_robertson_2025} also employs list experiments to estimate support for territorial concessions during wartime, finding that only a small minority of citizens are willing to ``trade land for peace'' and that opposition to concessions is largely genuine.

\section{Research Design and Methodology}

This study addresses the research question: ``How many Japanese citizens are genuinely indifferent to territorial disputes?'' An online, opt-in panel survey was conducted on June 28--29, 2025, with a sample of 4{,}500 Japanese citizens aged 18 to 99. Quotas were set for gender and region (47 prefectures) to enhance the representativeness of the sample.

Respondents were randomly assigned to either a control group or one of three treatment groups, each focused on a different territory (Takeshima, the Senkaku Islands, or the Northern Territories).
\footnote{N=1,147 for control; N=1,106 for Takeshima treatment; N=1,128 for Senkaku treatment; N=1,119 for Northern Territories treatment.}

\paragraph{Control Group.} Respondents were shown a list of four non-sensitive items and asked how many applied to them. The items were designed to be politically neutral and varied in their likely applicability to the general population:
\begin{enumerate}
  \item I would rather watch movies in a theater than at home.
  \item I tend to plan my trips in detail beforehand.
  \item I have made a donation(s) through the Furusato Nozei (Hometown Tax Donation) system before.
  \item I have purchased a physical book or magazine from a bookstore within the last month.
\end{enumerate}

\paragraph{Treatment Groups.} Respondents received the same four non-sensitive items plus one sensitive item. The sensitive item was phrased to capture indifference:
\begin{quote}
  Personally, I am not concerned about whether Japan or [South Korea / China / Russia] has sovereignty over [Takeshima / the Senkaku Islands / the Northern Territories].
  \footnote{\emph{Original: }\jptext{[竹島 / 尖閣諸島 / 北方領土]が日本のものか[韓国 / 中国 / ロシア]のものか、個人的にはこだわらない}}
\end{quote}

The difference in the mean of item counts between a treatment group and the control group provides the estimated proportion of respondents who agree with the sensitive item, that is, the proportion who are indifferent to that specific territorial dispute.

\section{Results}

\begin{table}[htbp]
\centering
\caption{Summary of estimated indifference toward territorial disputes}
\label{tab:summary_indifference}
\renewcommand{\arraystretch}{1.35} % ← 行間の倍率(1.0 がデフォルト)
\small
\setlength{\tabcolsep}{8pt} % 列間の余白を調整(ここを変えるとさらに広くできる)
% 行間の余白も調整

\begin{tabular}{l p{3.3cm} p{3.3cm} p{3.3cm}}
\toprule
& \textbf{Takeshima} & \textbf{Senkaku} & \textbf{Northern Territories} \\
\midrule
\textbf{Overall} &
\textbf{19\%} &
\textbf{11\%} &
\textbf{5\%} \\

\textbf{By region} &
Tokyo (48\%), Kyushu (39\%), Kinki (32\%) &
Hokuriku-Shinetsu (38\%), \newline Tokyo (26\%) &
Hokkaido (34\%), Hokuriku-Shinetsu (25\%) \\

\textbf{By age} &
40s (38\%) &
40s, 60s (24–25\%) &
20s (45\%) \\

\textbf{By employment} &
Non-regular (45\%) &
Non-regular (30\%) &
Self-employed (26\%) \\
\bottomrule
\end{tabular}
\renewcommand{\arraystretch}{1.0} % ← 必ず元に戻す
\end{table}

\subsection{Mean-in-differences}
\begin{figure}[t] % t: ページ上部, h: その場, b: 下部 など
  \centering
  \includegraphics[width=.9\textwidth]{figure_le_indifference_by_territory.pdf}
  \caption{Mean-in-differences estimates of latent indifference by territorial dispute}
  \label{fig:mid_territory}
\end{figure}

Figure~\ref{fig:mid_territory} presents the mean-in-differences estimates of latent indifference for each territorial dispute. The estimates indicate that approximately 19.8\% of respondents are indifferent to the sovereignty of Takeshima, 12.0\% to the Senkaku Islands, and 5.7\% to the Northern Territories. Estimates for Takeshima and the Senkaku Islands are statistically significant at the 0.05 level, while the estimate for the Northern Territories is not.

\subsection{Baseline estimates of latent indifference}

\begin{figure}[t] % t: ページ上部, h: その場, b: 下部 など
  \centering
  \includegraphics[width=1\textwidth]{figure_ICT_AME_indifference_by_territory.pdf}
  \caption{Estimated share indifferent by territorial dispute}
  \label{fig:ame_territory}
\end{figure}


Figure~\ref{fig:ame_territory} reports the average marginal effects (AMEs) of the treatment indicator for each territorial dispute, based on the fully interacted ICT regression model in Table~\ref{tab:ict_regression}. Substantively, these AMEs can be interpreted as the estimated percentage of respondents who agree with the statement that they are ``not concerned about which country has sovereignty'' over the respective disputed territory, holding the observed covariate distribution constant.

This model includes demographic covariates (gender, age, employment status) and regional indicators to control for potential confounders and improve estimate precision.

\begin{equation}
Y_i = \beta_0
    + \beta_1 \text{Treat}_i
    + \boldsymbol{\gamma}' \mathbf{X}_i
    + \boldsymbol{\delta}' (\text{Treat}_i \times \mathbf{X}_i)
    + \varepsilon_i,
\end{equation}
where $Y_i$ is the count of items selected by respondent $i$, $\text{Treat}_i$ is the treatment indicator, $\mathbf{X}_i$ is a vector of covariates, and $\varepsilon_i$ is the error term.
The vector $\boldsymbol{\delta}'$ captures how the treatment effect varies with the covariates. The AMEs reported in Figure~\ref{fig:ame_territory} and the subsequent marginal-effects figures are obtained by averaging the individual treatment effects, $\hat{Y}_i(\text{Treat}=1) - \hat{Y}_i(\text{Treat}=0)$, over the observed covariate distribution.

For the analysis, prefectures are grouped into eleven proportional representation (PR) electoral districts used in elections to the House of Representatives: Hokkaido, Tohoku, Kita-Kanto, Minami-Kanto, Tokyo, Hokuriku-Shinetsu, Tokai, Kinki, Chugoku, Shikoku, and Kyushu\footnote{ Table~\ref{tab:pr_blocks} provides the mapping of prefectures to PR blocks.}.

The estimates show that hidden indifference is far from negligible. For Takeshima, the AME is about 0.188, implying that roughly 19\% of Japanese citizens are indifferent to the sovereignty of Takeshima. For the Senkaku Islands, the estimated share of indifferent citizens is around 11\%, while for the Northern Territories the estimate is approximately 5\%. Although the latter is smaller in magnitude, it is still statistically significant, indicating the presence of a non-trivial group that does not hold strong views even on this highly salient dispute.

These figures contrast sharply with government opinion polls, in which large majorities report concern about the disputes when asked directly. The list experiment therefore reveals considerable latent heterogeneity in territorial attitudes that is masked by social desirability in conventional surveys.

\subsection{Heterogeneity by age}

\begin{figure}[t] % t: ページ上部, h: その場, b: 下部 など
  \centering
  \includegraphics[width=1\textwidth]{figure_ICT_marginal_age_by_territory.pdf}
  \caption{Estimated share indifferent by age group}
  \label{fig:marginal_age}
\end{figure}

Figure~\ref{fig:marginal_age} reports the marginal effect of the treatment indicator by age group for each territory. For Takeshima, indifference is relatively concentrated among middle-aged respondents: those in their 40s show the highest estimated latent indifference, with the 95\% confidence interval clearly above zero. At the same time, indifference is also observed throughout other age groups.
For the Senkaku Islands, the age pattern is less flat, but again respondents in their 40s and 60s show somewhat higher latent indifference than other groups.
In contrast, for the Northern Territories the pattern is more polarized: younger respondents in their 20s exhibit relatively high indifference, whereas older cohorts are less indifferent. This combination suggests that the Northern Territories have become more salient for older citizens, likely in the context of Russia's invasion of Ukraine, while younger citizens are comparatively less concerned with the issue.

Overall, the age profiles indicate that hidden indifference is not simply a ``youth apathy'' story. Rather, indifference peaks at different points in the age cohort spectrum across disputes, reflecting the interaction between generational experience and issue salience.

\subsection{Heterogeneity by region and employment}

\begin{figure}[t] % t: ページ上部, h: その場, b: 下部 など
  \centering
  \includegraphics[width=1\textwidth]{figure_ICT_marginal_PRblock_by_territory.pdf}
  \caption{Estimated share indifferent by PR block}
  \label{fig:marginal_block}
\end{figure}

Figure~\ref{fig:marginal_block} examines how latent indifference varies across proportional representation (PR) blocks. For Takeshima, indifference is particularly high in metropolitan and urbanized regions such as Tokyo, Kinki, as well as Hokuriku-Shinetsu and Kyushu. A similar pattern holds for the Senkaku Islands, where residents in Tokyo and Hokuriku-Shinetsu also show elevated levels of indifference.
At the same time, residents in geographically close areas (e.g.\ Chugoku for Takeshima; Kyushu for the Senkaku Islands) tend to be less indifferent, and in several cases the estimated marginal effects are small and statistically indistinguishable from zero. This pattern is consistent with the idea that physical proximity to the disputed areas increases the perceived relevance of territorial issues, thereby lowering indifference.

In contrast, the Northern Territories provide a more nuanced picture. While the national average indifference is relatively low, Hokkaido---the region geographically closest to the disputed islands---shows substantially higher indifference than most other blocks. One possible interpretation is that residents in Hokkaido may place more weight on pragmatic concerns in their everyday lives, such as local economic conditions, than on symbolic sovereignty claims, even under heightened tensions with Russia.

\begin{figure}[t] % t: ページ上部, h: その場, b: 下部 など
  \centering
  \includegraphics[width=1\textwidth]{figure_ICT_marginal_employment_by_territory.pdf}
  \caption{Estimated share indifferent by employment status}
  \label{fig:marginal_employment}
\end{figure}

Figure~\ref{fig:marginal_employment} further explores heterogeneity in the AMEs across employment status. For Takeshima and the Senkaku Islands, those in non-regular employment (part-time, temporary, contract workers) exhibit the highest levels of latent indifference, whereas self-employed respondents show the highest indifference for the Northern Territories.

These patterns suggest that precarious or non-standard labor market positions are associated with weaker attachment to territorial claims. One interpretation is that material insecurity and everyday economic concerns are more salient for these groups, diverting attention from political issues.

\subsection{Robustness checks}

Because list experiments are sensitive to design failures, I conduct several checks recommended in the literature.

First, I examine potential floor and ceiling effects, following \citet{blair2012}. Appendix Figure~\ref{fig:appendix_floor_ceiling} shows the distribution of item counts by territory and experimental group. In addition, I conduct simple boundary tests that compare (i) the probability of reporting zero items and (ii) the probability of reporting the maximum count between the control and treatment lists (Fisher's exact tests). Appendix Figure~\ref{fig:appendix_floor_ceiling_blair_imai} visualizes the treatment--control differences, and Appendix Table~\ref{tab:appendix_floor_ceiling_tests} reports the corresponding p-values. The probability of reporting the maximum is substantially lower in the treatment lists across all three disputes (all $p<10^{-5}$), consistent with mild ``edge avoidance'' behavior often attributed to privacy concerns. However, the overall mass at the boundaries is small (around 3\% at the ceiling in the control lists and below 1\% at the ceiling in the treatment lists), suggesting that floor/ceiling effects are unlikely to drive the main results.

Second, I conduct a placebo test. I construct a dataset that combines the control group with each treatment group and regress the count of non-sensitive items on a pseudo-treatment indicator and the full set of covariates. Because the sensitive item is not included in these counts, the treatment indicator should have no effect if the randomization and the design are valid. Consistent with this expectation, the treatment coefficient is effectively unidentified (and statistically null) once the model drops the collinear indicator, and the covariate patterns are stable across placebo specifications. The specification and placeholder results are reported in Appendix Table~\ref{tab:placebo_reg}. This strengthens confidence that the positive AMEs reported above are capturing genuine responses to the sensitive item, rather than artefacts of the survey design.


Third, I implement the formal design-effect test proposed by \citet{blair2012}, which evaluates a set of stochastic-dominance inequalities (their Eq.~19) that must hold if the sensitive item does not distort responses to the control items. Using a nonparametric within-group bootstrap to obtain p-values, the inequalities are not rejected for Takeshima and the Senkaku Islands, and although the Northern Territories exhibits a very small negative margin, the overall test remains far from rejection. Appendix Figure~\ref{fig:appendix_blair_imai_margins} and Appendix Table~\ref{tab:appendix_blair_imai_design_effect} report the results.

Fourth, I replicate the baseline prevalence estimates using the maximum-likelihood implementation in the \texttt{list} package (\texttt{ictreg}), which follows \citet{blair2012}. In practice, fully interacted specifications with many dummy variables can yield unstable Hessians in ML estimation; therefore I report the intercept-only ML estimates as a robustness check of the national average prevalence. The ML-based prevalence estimates closely track the difference-in-means results, and the posterior mean probabilities of endorsing the sensitive item (from \texttt{pred.post}) match the implied inv-logit prevalence.

In sum, these robustness checks provide reassurance that the list experiment was well designed and executed, and that the estimated levels of latent indifference are not driven by methodological artifacts.


\section{Discussion}

\begin{table}[t]
\centering
\caption{Summary of interpretation across territorial disputes}
\label{tab:interpretation_summary}
\renewcommand{\arraystretch}{1.35} % ← 行間の倍率(1.0 がデフォルト)
\small
\setlength{\tabcolsep}{8pt} % 列間の余白を調整(ここを変えるとさらに広くできる)
\begin{tabular}{
  p{2.5cm}    % 行ラベル
  p{4cm}  % Takeshima
  p{4cm}  % Senkaku
  p{4cm}  % Northern Territories
}
\toprule
& \textbf{Takeshima} & \textbf{Senkaku Islands} & \textbf{Northern Territories} \\
\midrule

\textbf{Overall pattern} &
Hidden indifference is most prevalent. &
Moderate level of hidden indifference. &
More concerned, likely reflecting the international environment. \\[0.9em]

\textbf{Geographic pattern} &
More indifferent in metropolitan areas, \newline
less indifferent in geographically close areas. &
Indifference concentrated in Hokuriku-Shinetsu and Tokyo. &
More indifferent in geographically close area (e.g.\ Hokkaido). \\[0.9em]

\textbf{Demographic groups} &
Those in their 40s; non-regular workers tend to be less concerned. &
Those in their 40s and 60s; non-regular workers tend to be less concerned. &
Those in their 20s; self-employed workers tend to be less concerned. \\

\bottomrule
\end{tabular}
\renewcommand{\arraystretch}{1.0} % ← 行間の倍率(1.0 がデフォルト)
\end{table}

The list experiment reveals that a sizable portion of the Japanese public harbors latent indifference toward territorial disputes. Table~\ref{tab:interpretation_summary} summarizes the key patterns observed across the three disputes. Roughly one in five respondents are estimated to be indifferent to the sovereignty of Takeshima, about one in nine to the Senkaku Islands, and around one in twenty to the Northern Territories.

Indifference is not randomly distributed. It is systematically structured by geographical and socioeconomic factors in ways that are broadly consistent with existing theories of territorial politics and political behavior.

For Takeshima, hidden indifference is most pronounced.
Indifference is especially high in metropolitan and urban regions such as Tokyo and Kinki, as well as in Hokuriku-Shinetsu and Kyushu. By contrast, residents of geographically close areas such as Chugoku show lower levels of indifference. Age-wise, respondents in their 40s show the highest latent indifference, and non-regular workers are also more likely to be indifferent. This is striking given the high symbolic salience of Takeshima in Japan--South Korea relations and its prominence in nationalist rhetoric. One plausible interpretation is that direct surveys overstate public concern because respondents feel pressure to express patriotic commitment, even though a substantial minority are privately willing to live with ambiguity over sovereignty.

The pattern for the Senkaku Islands is similar in several respects. Approximately 11\% of respondents are estimated to be indifferent, with indifference particularly pronounced among residents of Tokyo and Hokuriku-Shinetsu, and lower among those in geographically closer areas such as Kyushu.
Indifference is somewhat higher among respondents in their 40s and 60s, and again among non-regular workers.
Given that the Senkaku dispute is embedded in a broader security rivalry with China, this is notable: even in a highly securitized dispute, a sizable share of the public does not hold strong views on sovereignty once social desirability pressures are removed.

By contrast, the Northern Territories exhibit the lowest overall level of hidden indifference, at around 5\%.
Nationally, this is consistent with heightened salience of Russia due to the ongoing invasion of Ukraine. Yet the regional pattern is more nuanced. Hokkaido, the region geographically closest to the disputed islands and most directly affected by cross-border interactions, shows substantially higher indifference than most other blocks.
Younger respondents in their 20s are also more likely to be indifferent, and self-employed individuals show relatively elevated indifference.
These patterns suggest that local pragmatic concerns and economic priorities may temper symbolic attachment to sovereignty, even when national-level rhetoric stresses threat and indivisibility.

Taken together, these results complicate the standard picture of territorial disputes as issues backed by uniformly hawkish publics.
Even on highly salient disputes, a sizable minority of citizens are quietly indifferent. Moreover, indifference is not randomly distributed but appears to be conditioned by geographic proximity and economic security: those in precarious labor-market positions, or facing more immediate economic constraints, are less likely to engage in symbolic claims over distant islands. This matches broader findings that economic insecurity lowers political engagement.

Crucially, this ``hidden indifference'' should not be interpreted as active support for territorial concessions. The framing of the sensitive item---``personally not concerned''---captures a lack of issue salience rather than a compromising policy preference. Respondents facing economic insecurity or living far from the disputed areas likely rank territorial sovereignty low on their hierarchy of needs compared to immediate concerns. This suggests that the indifferent group is characterized by political apathy stemming from material constraints, rather than a cosmopolitan or pacifist ideology. They are not necessarily demanding that the government give up the islands; rather, they are unlikely to punish the government for prioritizing pragmatic solutions over symbolic sovereignty.

The findings also have implications for understanding the two-level game dynamic in Japanese foreign policy. On the one hand, governments can and do mobilize territorial disputes to generate nationalist sentiment and constrain diplomatic flexibility. On the other hand, the presence of a substantial indifferent group, especially among younger and economically vulnerable citizens, implies that domestic public opinion may be less hardline than official surveys suggest. If politicians recognize that public opinion is diversified, there might be more room to explore arrangements that de-emphasize formal sovereignty while addressing local economic and security concerns.

At the same time, the concentration of indifference among precarious workers raises normative questions about whose preferences are most likely to be represented in the foreign-policy process.
Citizens who are most insulated from economic insecurity may also be those most invested in symbolic territorial claims, potentially skewing the perceived ``national interest'' toward more hawkish positions.
Future research should investigate the stability of these indifferent attitudes over time and their responsiveness to elite framing, as well as triangulate list experiments with qualitative interviews and panel data.

\section{Conclusion}

This paper employs a list experiment to estimate the percentage of Japanese citizens who are indifferent to three territorial disputes that Japan is involved in. The indirect estimates reveal that roughly 19\% of citizens are indifferent to the sovereignty of Takeshima, 11\% to the Senkaku Islands, and 5\% to the Northern Territories. These attitudes are not randomly distributed, but vary systematically by region, age, and employment status.

Methodologically, the study demonstrates that indirect techniques such as list experiments can uncover structure in territorial attitudes that is invisible in direct questioning. Substantively, the findings suggest that domestic constraints on diplomatic compromise may be more flexible than conventional surveys imply, particularly among younger cohorts and those in precarious job status.

Several limitations remain. The data come from a single cross-sectional online survey, which limits the ability to track changes over time or to examine causal responses to elite framing and geopolitical events. The list experiment cannot distinguish between different reasons for indifference, such as fatigue, distrust, or cosmopolitan orientations. Future work could address these limitations by combining list experiments with panel surveys, conjoint experiments, or in-depth interviews.

Despite these limitations, the results highlight the importance of moving beyond headline figures of ``concern'' in public opinion on territorial disputes.
Beneath the surface of apparently unified national sentiment lies a heterogeneous public in which a substantial minority are quietly indifferent.
Recognizing this heterogeneity is crucial for understanding the domestic foundations of Japanese foreign policy and the possibilities for more flexible and pragmatic approaches to long-standing territorial conflicts.

\section*{Acknowledgments}

I thank JX Press Corporation for providing access to the survey data used in this study. The views expressed here are my own and do not represent those of the company.

\section*{Data Availability}
The survey data used in this paper were provided by JX Press Corporation and cannot be publicly shared due to contractual and confidentiality restrictions. Replication code using simulated example data is available upon request.

\section*{Conflict of Interest}
The author has an outsourcing contract with JX Press Corporation, which provided access to the data used in this study. The company had no role in the design, analysis, or conclusions of this research.


\bibliography{references}

\newpage
\appendix
\section*{Appendix}
\addcontentsline{toc}{section}{Appendix}

% Appendix 図表を A1, A2, ... 表記にする場合(推奨)
\renewcommand{\thetable}{A\arabic{table}}
\renewcommand{\thefigure}{A\arabic{figure}}
\setcounter{table}{0}
\setcounter{figure}{0}

\section{Full ICT Regression Results}
\begin{table}[htbp]
\centering
\caption{ICT Regression Results for Three Territorial Disputes (Key Coefficients)}
\label{tab:ict_regression}
\begin{threeparttable}
\scriptsize % small → scriptsize に変更
\setlength{\tabcolsep}{2pt} % 列の左右の余白を狭くする
\renewcommand{\arraystretch}{1.1} % 行間を詰める
\begin{tabular}{lccc}
\toprule
& Takeshima & Senkaku & Northern Territories \\
\midrule
Treatment (List) & \num{-0.120} & \num{0.235} & \num{0.348} \\
& (\num{0.296}) & (\num{0.316}) & (\num{0.305}) \\
Treatment × Male & \num{-0.157} & \num{-0.175}+ & \num{-0.152} \\
& (\num{0.096}) & (\num{0.093}) & (\num{0.093}) \\
Treatment × Other gender & \num{-0.565} & \num{-0.309} & \num{0.005} \\
& (\num{0.387}) & (\num{0.397}) & (\num{0.396}) \\
Treatment × Age: 30s & \num{-0.010} & \num{0.057} & \num{-0.634}* \\
& (\num{0.263}) & (\num{0.261}) & (\num{0.265}) \\
Treatment × Age: 40s & \num{0.165} & \num{0.092} & \num{-0.528}* \\
& (\num{0.252}) & (\num{0.254}) & (\num{0.257}) \\
Treatment × Age: 50s & \num{-0.152} & \num{-0.205} & \num{-0.656}* \\
& (\num{0.251}) & (\num{0.249}) & (\num{0.256}) \\
Treatment × Age: 60s & \num{-0.075} & \num{0.097} & \num{-0.323} \\
& (\num{0.256}) & (\num{0.254}) & (\num{0.260}) \\
Treatment × Age: 70+ & \num{-0.074} & \num{-0.124} & \num{-0.408} \\
& (\num{0.249}) & (\num{0.248}) & (\num{0.255}) \\
Treatment × Hokkaido & \num{0.498}+ & \num{0.146} & \num{0.545}+ \\
& (\num{0.286}) & (\num{0.292}) & (\num{0.285}) \\
Treatment × Tohoku & \num{0.435} & \num{0.011} & \num{0.159} \\
& (\num{0.278}) & (\num{0.299}) & (\num{0.280}) \\
Treatment × Kita-Kanto & \num{0.339} & \num{-0.073} & \num{0.202} \\
& (\num{0.244}) & (\num{0.260}) & (\num{0.239}) \\
Treatment × Minami-Kanto & \num{0.292} & \num{-0.018} & \num{0.251} \\
& (\num{0.232}) & (\num{0.250}) & (\num{0.227}) \\
Treatment × Tokyo & \num{0.710}** & \num{0.133} & \num{0.356} \\
& (\num{0.238}) & (\num{0.258}) & (\num{0.234}) \\
Treatment × Hokuriku-Shinetsu & \num{0.558}* & \num{0.263} & \num{0.449} \\
& (\num{0.279}) & (\num{0.299}) & (\num{0.275}) \\
Treatment × Tokai & \num{0.263} & \num{-0.049} & \num{0.101} \\
& (\num{0.241}) & (\num{0.258}) & (\num{0.235}) \\
Treatment × Kinki & \num{0.579}* & \num{-0.044} & \num{0.383}+ \\
& (\num{0.227}) & (\num{0.245}) & (\num{0.219}) \\
Treatment × Shikoku & \num{0.139} & \num{-0.339} & \num{-0.055} \\
& (\num{0.356}) & (\num{0.383}) & (\num{0.339}) \\
Treatment × Kyushu & \num{0.556}* & \num{-0.054} & \num{0.285} \\
& (\num{0.255}) & (\num{0.275}) & (\num{0.253}) \\
\midrule
Num.Obs. & \num{2253} & \num{2275} & \num{2266} \\
R2       & \num{0.038} & \num{0.026} & \num{0.033} \\
F        & \num{2.469} & \num{1.690} & \num{2.154} \\
RMSE     & \num{1.10}  & \num{1.09}  & \num{1.08} \\
\bottomrule
\end{tabular}
\begin{tablenotes}[flushleft]
\footnotesize
\item + p $<$ 0.1, * p $<$ 0.05, ** p $<$ 0.01, *** p $<$ 0.001
\item OLS models of list experiment item counts. Coefficients on Treatment and its interactions capture latent indifference; coefficients without Treatment are omitted for brevity.
\end{tablenotes}
\end{threeparttable}
\end{table}

\FloatBarrier

\section{Robustness Checks}


\subsection{Floor and Ceiling Effects}

\begin{center}
  \includegraphics[width=0.75\textwidth]{figure_ICT_hist_Y_by_territory_group.pdf}
  \captionof{figure}{Distribution of item counts by territory and experimental group}
  \label{fig:appendix_floor_ceiling}
\end{center}

\bigskip

\begin{center}
  \includegraphics[width=1\textwidth]{figure_floor_ceiling_blair_imai.pdf}
  \captionof{figure}{Floor/ceiling diagnostics (treatment minus control) following Blair and Imai (2012)}
  \label{fig:appendix_floor_ceiling_blair_imai}
\end{center}

\begin{table}[t]
  \centering
  \caption{Floor/ceiling boundary tests (Fisher's exact tests)}
  \label{tab:appendix_floor_ceiling_tests}
  \footnotesize
  \setlength{\tabcolsep}{4pt}  % Reduce column padding for compact layout
  \sisetup{
    table-number-alignment=center,
    tight-spacing=true
  }
  \begin{tabular}{
    l                                                   % Territory
    S[table-format=1.0]                                 % J
    S[table-format=4.0]                                 % N_0
    S[table-format=4.0]                                 % N_1
    S[table-format=1.3]                                 % Pr(Y=0|T=0)
    S[table-format=1.3]                                 % Pr(Y=0|T=1)
    S[table-format=1.3]                                 % p (floor)
    S[table-format=1.3]                                 % Pr(Y=J|T=0)
    S[table-format=1.3]                                 % Pr(Y=J+1|T=1)
    S[table-format=1.1e-1,scientific-notation=true]     % p (ceiling)
  }
    \toprule
    {Territory} & {$J$} & {$N_0$} & {$N_1$} &
    {$P_0^{\text{floor}}$} & {$P_1^{\text{floor}}$} & {$p_{\text{fl}}$} &
    {$P_0^{\text{ceil}}$} & {$P_1^{\text{ceil}}$} & {$p_{\text{cl}}$} \\
    \midrule
    Takeshima            & 4 & 1147 & 1106 & 0.340 & 0.290 & 0.011 & 0.031 & 0.006 & 8.5e-6  \\
    Senkaku              & 4 & 1147 & 1128 & 0.340 & 0.295 & 0.024 & 0.031 & 0.004 & 1.5e-7  \\
    Northern Territories & 4 & 1147 & 1119 & 0.340 & 0.329 & 0.593 & 0.031 & 0.004 & 6.8e-7  \\
    \bottomrule
  \end{tabular}

  \begin{threeparttable}
  \begin{tablenotes}[flushleft]
    \footnotesize
    \item Notes: $J$ is the number of non-sensitive control items. $N_0$ and $N_1$ denote sample sizes in the control and treatment groups. ``Floor'' compares $\Pr(Y=0\mid T=0)$ and $\Pr(Y=0\mid T=1)$. ``Ceiling'' compares the probability of the maximum count in each list: $\Pr(Y=J\mid T=0)$ for the control list and $\Pr(Y=J{+}1\mid T=1)$ for the treatment list. Following \citet{blair2012}, a substantially lower ceiling probability in the treatment list is consistent with mild edge-avoidance behavior.
  \end{tablenotes}
  \end{threeparttable}
\end{table}

\clearpage
\subsection{Design-Effect Check}

\begin{center}
  \includegraphics[width=1\textwidth]{figure_design_effect_blair_imai_margins.pdf}
  \captionof{figure}{Design-effect check using the Blair and Imai (2012) inequality formulation (Eq.~19)}
  \label{fig:appendix_blair_imai_margins}
\end{center}

\begin{table}[htbp]
  \centering
  \caption{Blair and Imai (2012) design-effect test results (bootstrap)}
  \label{tab:appendix_blair_imai_design_effect}
  \small
  \begin{tabular}{lrrrrr}
    \toprule
    Territory & $J$ & Min margin & $\min(A)$ & $\min(B)$ & Bootstrap $p$ \\
    \midrule
    Takeshima & 4 & 0.0084 & 0.0084 & 0.0251 & 0.5015 \\
    Senkaku & 4 & 0.0041 & 0.0041 & 0.0278 & 0.5240 \\
    Northern Territories & 4 & -0.0082 & -0.0082 & 0.0269 & 0.5920 \\
    \bottomrule
  \end{tabular}

  \begin{threeparttable}
  \begin{tablenotes}[flushleft]
    \footnotesize
    \item Notes: $J$ is the number of non-sensitive (control) items. The test evaluates the set of stochastic-dominance inequalities in \citet[][Eq.~19]{blair2012}. ``Min margin'' is the most negative (i.e., worst) inequality margin across all constraints; non-negative values are consistent with no design effects. Bootstrap p-values are computed via within-group resampling.
  \end{tablenotes}
  \end{threeparttable}
\end{table}


% Start the placebo subsection on a fresh page and keep its heading with the regression table
\clearpage
\subsection{Placebo Regression (pseudo-treatment)}

% Ensure the regression table is placed on the same page as the subsection header
\begingroup
\renewcommand{\floatpagefraction}{.9}
\renewcommand{\topfraction}{.9}
\renewcommand{\bottomfraction}{.9}
\renewcommand{\textfraction}{.1}
\begin{table}[htbp]
\centering
\caption{Placebo Regression Using Control-Item Counts}
\label{tab:placebo_reg}
\begin{threeparttable}
\scriptsize % small → scriptsize に変更
\renewcommand{\arraystretch}{1.00} % 行間を詰める
\begin{tabular*}{\textwidth}{@{\extracolsep{\fill}}lc}
\toprule
& Placebo \\
\midrule
Intercept & \num{1.224}*** \\
& (\num{0.205}) \\
Male & \num{0.082} \\
& (\num{0.072}) \\
Other gender & \num{-0.075} \\
& (\num{0.267}) \\
Age: 30s & \num{0.201} \\
& (\num{0.176}) \\
Age: 40s & \num{0.075} \\
& (\num{0.170}) \\
Age: 50s & \num{0.190} \\
& (\num{0.168}) \\
Age: 60s & \num{0.249} \\
& (\num{0.173}) \\
Age: 70+ & \num{0.373}* \\
& (\num{0.174}) \\
Hokkaido & \num{-0.080} \\
& (\num{0.185}) \\
Tohoku & \num{-0.185} \\
& (\num{0.182}) \\
Kita-Kanto & \num{0.058} \\
& (\num{0.154}) \\
Minami-Kanto & \num{0.136} \\
& (\num{0.147}) \\
Tokyo & \num{-0.045} \\
& (\num{0.153}) \\
Hokuriku-Shinetsu & \num{-0.200} \\
& (\num{0.179}) \\
Tokai & \num{0.055} \\
& (\num{0.155}) \\
Kinki & \num{-0.014} \\
& (\num{0.142}) \\
Shikoku & \num{0.272} \\
& (\num{0.246}) \\
Kyushu & \num{-0.080} \\
& (\num{0.162}) \\
Non-regular & \num{-0.515}*** \\
& (\num{0.103}) \\
Self-employed & \num{-0.244}+ \\
& (\num{0.133}) \\
Not working & \num{-0.509}*** \\
& (\num{0.091}) \\
Other employment & \num{-0.327}** \\
& (\num{0.117}) \\
\midrule
Num.Obs. & \num{1147} \\
R2 & \num{0.060} \\
\bottomrule
\end{tabular*}
\begin{tablenotes}[flushleft]
\footnotesize
\item + p $<$ 0.1, * p $<$ 0.05, ** p $<$ 0.01, *** p $<$ 0.001.
\item Dependent variable: number of non-sensitive control items (0--4).
\item The pseudo-treatment indicator is perfectly collinear with the intercept and therefore omitted.
\item The same control group is used for all disputes; hence a single-column placebo test is sufficient.
\end{tablenotes}
\end{threeparttable}
\end{table}
\endgroup

% Prevent the next section from floating up above the placebo table
\FloatBarrier
\clearpage
\section{Regional Coding: Prefectures and PR Blocks}

Table~\ref{tab:pr_blocks} shows how Japan's 47 prefectures are grouped into
the eleven proportional representation (PR) electoral districts used in the analysis.

\begin{table}[htbp]
  \centering
  \caption{Mapping of prefectures to PR electoral districts}
  \label{tab:pr_blocks}
  \small
  \begin{tabular}{ll}
    \toprule
    PR block            & Prefectures \\
    \midrule
    Hokkaido            & Hokkaido \\
    Tohoku              & Aomori, Iwate, Miyagi, Akita, Yamagata, Fukushima \\
    Kita-Kanto          & Ibaraki, Tochigi, Gunma, Saitama \\
    Minami-Kanto        & Chiba, Kanagawa, Yamanashi \\
    Tokyo               & Tokyo \\
    Hokuriku-Shinetsu   & Niigata, Toyama, Ishikawa, Fukui, Nagano \\
    Tokai               & Gifu, Shizuoka, Aichi, Mie \\
    Kinki               & Shiga, Kyoto, Osaka, Hyogo, Nara, Wakayama \\
    Chugoku             & Tottori, Shimane, Okayama, Hiroshima, Yamaguchi \\
    Shikoku             & Tokushima, Kagawa, Ehime, Kochi \\
    Kyushu              & Fukuoka, Saga, Nagasaki, Kumamoto, Oita, Miyazaki, Kagoshima, Okinawa \\
    \bottomrule
  \end{tabular}
\end{table}

\end{document}