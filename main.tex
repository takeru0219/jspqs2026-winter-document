\documentclass[12pt,a4paper]{article}
\pagestyle{plain}

% ---------- 基本レイアウト ----------
\usepackage[margin=25mm]{geometry}
\usepackage{setspace}
\onehalfspacing

% ---------- フォント・文字コード ----------
\usepackage[T1]{fontenc}
\usepackage[utf8]{inputenc}
\usepackage{lmodern} % 必要なら newtxtext,newtxmath に変更可

% ---------- 日本語サポート ----------
\usepackage{CJKutf8} % 日本語表示用
\newcommand{\jptext}[1]{\begin{CJK}{UTF8}{ipxm}#1\end{CJK}} % IPAex明朝フォント使用

% ---------- 数式・図表 ----------
\usepackage{amsmath,amssymb}
\usepackage{graphicx}
\usepackage{booktabs}
\usepackage{siunitx}
\usepackage{threeparttable}
\usepackage{caption}
\captionsetup{labelfont=bf}

\usepackage{placeins} % \FloatBarrier コマンド用

% ---------- ハイパーリンク ----------
\usepackage[hidelinks]{hyperref}

% ---------- 参考文献(APSA風) ----------
\usepackage[authoryear]{natbib}
\bibliographystyle{apsr}

% ---------- タイトル情報 ----------
\title{Beneath the Surface: Measuring Hidden Indifference\\
in Japanese Territorial Attitudes}

\author{Takeru Eto\thanks{JX Press Corporation / Graduate School of Arts and Sciences, The University of Tokyo. Email:
\texttt{takeru0219@gmail.com}.}}

\date{Draft: \today}

\begin{document}

\maketitle

\begin{abstract}
Japan is currently engaged in three major territorial disputes: the Northern Territories
(known as the Kuril Islands to Russia), Takeshima (Liancourt Rocks, or Dokdo to South
Korea), and the Senkaku Islands (Diaoyu to China and Tiaoyutai to Taiwan).
Official government surveys consistently report high levels of public concern regarding these
territories. However, public opinion on matters of national identity and territory is not
always straightforward, and respondents may overstate their concern due to social
desirability bias. Expressing indifference on a matter of national sovereignty could be
perceived as unpatriotic, leading respondents to conceal their genuine lack of a strong
opinion.

To address this issue, this paper employs a list experiment embedded in a large-scale online
survey of 4{,}500 Japanese citizens. The design estimates the proportion of the population
that is privately indifferent to each territorial dispute. The results reveal substantial levels of
hidden indifference: approximately 19\% of respondents do not have a strong opinion on the
sovereignty of Takeshima, around 11\% are indifferent to the Senkaku Islands dispute, and around 5\%
do not have strong opinion on the Northern Territories.
These findings challenge the conventional view of a uniformly concerned public and
suggest that the domestic political constraints facing Japanese leaders may be more flexible
than commonly assumed.
\end{abstract}

\bigskip
\noindent\textbf{Keywords:} territorial disputes; public opinion; social desirability bias; list experiment; Japan

\section{Introduction}

Japan is currently engaged in three major territorial disputes: the Northern Territories (known
as the Kuril Islands to Russia), Takeshima (Liancourt Rocks, or Dokdo to South Korea), and
the Senkaku Islands (Diaoyu to China and Tiaoyutai to Taiwan) \citep{bukh2018}. Official
government surveys consistently report high levels of public concern regarding these
territories. For instance, a 2023 survey by the Cabinet Office found that 63.6\% of respondents
were ``concerned'' about the Takeshima issue \citep{cabinet2023takeshima}, while a 2024
survey reported that 78.4\% were concerned about the Senkaku Islands
\citep{cabinet2024senkaku}. These figures are often cited as evidence of a strong, unified
national consensus demanding a firm governmental stance.

However, public opinion on matters of national identity and territory is not always
straightforward. Governments may leverage territorial issues to stoke nationalist sentiment
for domestic political gain \citep{fang2022}, as has been argued in the case of Japan's policy
on the Northern Territories \citep{bukh2018}. Conversely, strong public sentiment can
constrain a government's diplomatic flexibility, narrowing the ``zone of possible agreement''
by making any perceived concession politically costly \citep{loizides2022}. Given these
dynamics, a critical question arises: Are the Japanese people truly as concerned with these
territorial disputes as direct surveys suggest?

This paper posits that conventional surveys may overestimate public engagement due
to social desirability bias (SDB), the tendency of respondents to give answers they believe are
socially acceptable rather than expressing their true beliefs. Expressing indifference on a
matter of national sovereignty could be perceived as unpatriotic, leading respondents to
conceal their genuine lack of a strong opinion. Moreover, as these surveys are conducted by
the Japanese government, they inquire whether respondents are concerned about each issue
and what actions they believe the Japanese government should undertake, without eliciting
respondents' claims or opinions regarding each disputed territory.

To overcome this methodological challenge, this study employs a list experiment, an
indirect questioning technique designed to elicit more truthful answers on sensitive topics
\citep{blair2012}. By embedding this experiment within a large-scale online survey, this
research provides a quantitative estimate of the proportion of the Japanese public that is
privately indifferent to these territorial disputes. The findings suggest that a considerable
portion of the population does not hold a strong opinion, revealing a more nuanced and
complex reality of public attitudes than is commonly assumed.

\section{Literature Review: Nationalism, Public Opinion, and Sensitive Attitudes}

\subsection{Public Opinion and Foreign Policy in Territorial Disputes}

The link between public opinion and foreign policy is a central theme in international
relations. \citet{putnam1988twolevel} argues that diplomatic negotiations should be understood as
``two-level games,'' which involve interactions between domestic and international factors.
In democratic states, leaders must remain attentive to public sentiment, as it can
impose significant political costs on those who are seen as compromising on core national
interests. Territorial disputes are particularly potent in this regard, as land is often framed as
an “indivisible’’ part of the national identity. Research shows that public opinion can drive
governments toward harder stances and limit their ability to negotiate peaceful resolutions
\citep{fang2022,loizides2022}.

However, the causal relationship is not always unidirectional. Governments also
actively shape public opinion. \citet{bukh2018} argues that the Japanese government
historically leveraged the Northern Territories issue as a tool to foster anti-Soviet sentiment,
a strategy that proved successful in domestic politics but ultimately hindered diplomatic
progress. This highlights the complex interplay between elite-led narratives and grassroots
public sentiment. While official discourse in Japan portrays a nation united in its concern for
its territories, no academic research has systematically investigated the “true’’ underlying
attitudes of the public, free from the pressures of social desirability.

Prior research has shown that public attitudes toward territorial disputes are far from uniform.
In particular, \citet{tanaka_2015} demonstrates that proximity to disputed areas and
economic considerations systematically shape citizens’ degree of hawkishness. His findings
highlight that even in highly salient territorial conflicts, sub-national heterogeneity exists,
challenging the assumption of a monolithic nationalist public. These insights imply that
variation may exist not only in how strongly people feel about a dispute but also in whether
they hold a strong opinion at all—an issue that direct surveys are ill-equipped to detect.

Existing work also highlights the emotional mechanisms through which
territorial threats shape political attitudes \citep{kobayashi_katagiri_2018}. Threats from rival
states can evoke anger and heighten nationalist pressure, thereby making expressions of
indifference socially costly. The current findings therefore suggest a parallel mechanism:
external threats may not only mobilize anger but also create social pressure to express
concern, prompting indifferent citizens to conceal their true attitudes.

\subsection{Measuring Sensitive Attitudes: The List Experiment}

Directly asking about sensitive topics—such as racial prejudice, corruption, or support for
extremist ideologies—is notoriously difficult due to social desirability bias
\citep{blair2012}. Respondents may mask their true views to avoid social sanction. This
phenomenon has been observed in various political contexts, from citizens concealing dissent
against the war in Ukraine to Taiwanese citizens hiding pro-unification attitudes
\citep{kramon2019,chapkovski2022,wu2024}.

The list experiment, also known as the Item Count Technique (ICT), offers a solution.
This indirect questioning method presents two randomly assigned groups with a list of items.
The control group receives non-sensitive “control’’ items, while the treatment group receives
the same list plus one sensitive item. Respondents report only the total number of applicable
items, not which ones. By providing this veil of privacy, the technique encourages more
honest responses and enables researchers to detect attitudes that remain hidden in direct
surveys \citep{blair2012,glynn2013,kramon2019}.

Consistent with this, \citet{kim_2015} finds weak demographic predictors of territorial
attitudes using direct surveys. The present study suggests one reason for such weak
associations: a large portion of respondents may be giving socially desirable answers,
masking true variation in latent attitudes. Indirect methods thus uncover structure that direct
questioning fails to detect, providing a more accurate picture of the distribution of public
opinion on territorial issues.

While recent work has begun to employ list experiments to study attitudes toward territorial concessions in Ukraine,
no study to date has systematically examined hidden indifference toward multiple territorial disputes
in a non-war setting such as contemporary Japan.
In the context of Russia’s invasion of Ukraine, \citet{daniels2025} uses a list experiment
to assess Ukrainians’ willingness to concede territory—specifically, Crimea.
The study finds that the estimated share supporting concessions is very small (around 5–6 percent)
and closely matches direct-question estimates, suggesting limited social desirability bias in that particular wartime context.
\citet{popeleches_robertson_2025} also employs list experiments to gauge support
for territorial concessions during wartime, finding that only a small minority of
citizens are willing to “trade land for peace” and that opposition to concessions is largely
genuine rather than a product of social desirability bias.

\section{Research Design and Methodology}

This study addresses the research question: ``How many Japanese citizens are genuinely
indifferent to the country's territorial disputes?'' An online, opt-in panel survey was
conducted on June 28--29, 2025, with a sample of 4{,}500 Japanese citizens aged 18 to 99.
Quotas were set for gender and region (47 prefectures) to enhance the representativeness of
the sample.

Respondents were randomly assigned to either a control group or one of three
treatment groups, each focused on a different territory (Takeshima, the Senkaku Islands, or
the Northern Territories).

\paragraph{Control Group.} Respondents were shown a list of four non-sensitive items and
asked how many applied to them. The items were designed to be politically neutral and
varied in their likely applicability to the general population:
\begin{enumerate}
  \item I would rather watch movies in a theater than at home.
  \item I tend to plan my trips in detail beforehand.
  \item I have made a donation(s) through the Furusato Nozei (Hometown Tax Donation) system before.
  \item I have purchased a physical book or magazine from a bookstore within the last month.
\end{enumerate}

\paragraph{Treatment Groups.} Respondents received the same four non-sensitive items plus
one sensitive item. The sensitive item was phrased to capture indifference:
\begin{quote}
  I am not concerned about which country has sovereignty over [Takeshima / the Senkaku Islands / the Northern Territories].
  \footnote{\emph{Original: }\jptext{[竹島 / 尖閣諸島 / 北方領土]が日本のものか[韓国 / 中国 / ロシア]のものか、個人的にはこだわらない}}
\end{quote}

The difference in the mean count of items between a treatment group and the control
group provides the estimated proportion of respondents who agree with the sensitive
item---that is, the proportion who are indifferent to that specific territorial dispute.

\section{Results}

\subsection{Mean-in-differences}
\begin{figure}[t] % t: ページ上部, h: その場, b: 下部 など
  \centering
  \includegraphics[width=.9\textwidth]{figure_le_indifference_by_territory.pdf}
  \caption{Mean-in-differences estimates of latent indifference by territorial dispute}
  \label{fig:ame_territory}
\end{figure}


\subsection{Baseline estimates of latent indifference}

\begin{figure}[t] % t: ページ上部, h: その場, b: 下部 など
  \centering
  \includegraphics[width=.9\textwidth]{figure_ICT_AME_indifference_by_territory.pdf}
  \caption{Estimated share indifferent by territorial dispute}
  \label{fig:ame_territory}
\end{figure}

Figure~\ref{fig:ame_territory} reports the average marginal effects (AMEs) of the
treatment indicator for each territorial dispute, based on the fully interacted ICT
regression model in Table~\ref{tab:ict_regression}. Substantively, these AMEs can be
interpreted as the estimated share of respondents who would agree with the indirect
statement that they are ``not concerned about which country has sovereignty'' over the
respective territory, holding the observed covariate distribution constant.

In addition to demographic characteristics (gender, age, and employment status),
the survey records respondents' prefecture of residence.
For the analysis, I group prefectures into eleven
proportional representation (PR) electoral districts used in elections to the House of
Representatives: Hokkaido, Tohoku, Kita-Kanto, Minami-Kanto, Tokyo, Hokuriku-Shinetsu,
Tokai, Kinki, Chugoku, Shikoku, and Kyushu\footnote{ Table~\ref{tab:pr_blocks} provides the mapping of prefectures to PR blocks.}.

The estimates show that hidden indifference is far from negligible. For Takeshima,
the AME is about 0.19, implying that roughly 19\% of Japanese citizens are indifferent
to the sovereignty of the Liancourt Rocks. For the Senkaku Islands, the estimated
share of indifferent citizens is around 11\%, while for the Northern Territories the
estimate is approximately 5\%. Although the latter is smaller in magnitude, it is still
statistically distinguishable from zero at conventional levels, indicating the
presence of a non-trivial group that does not hold strong views even on this highly
salient dispute.

These figures contrast sharply with government opinion polls, in which large
majorities report concern about the disputes when asked directly. The list experiment
therefore reveals considerable latent heterogeneity in territorial attitudes that is
masked by social desirability in conventional surveys.

\subsection{Heterogeneity by age}

\begin{figure}[t] % t: ページ上部, h: その場, b: 下部 など
  \centering
  \includegraphics[width=0.9\textwidth]{figure_ICT_marginal_age_by_territory.pdf}
  \caption{Estimated share indifferent by territorial dispute}
  \label{fig:marginal_age}
\end{figure}

Figure~\ref{fig:marginal_age} plots the marginal effect of the treatment indicator by
age group for each territory. For Takeshima, indifference is relatively concentrated
among middle-aged respondents: those in their 40s display the highest estimated
latent indifference, with the 95\% confidence interval clearly above zero. Indifference
is also non-trivial among respondents in their 30s and 50s, but the estimates for the
youngest and oldest cohorts are more modest and often statistically indistinguishable
from zero.

For the Senkaku Islands, the age pattern is flatter, but again respondents in their
40s and 60s show somewhat higher latent indifference than other groups. In contrast,
for the Northern Territories the pattern is more polarized: younger respondents in
their 20s exhibit relatively high indifference, whereas older cohorts are either less
indifferent or display wide intervals that include zero. This combination suggests that
the Northern Territories have become more salient for older citizens---likely in the
context of Russia's invasion of Ukraine---while younger citizens are comparatively
less engaged with the issue.

Overall, the age profiles indicate that hidden indifference is not simply a ``youth
apathy'' story. Rather, indifference peaks at different points in the life cycle across
disputes, reflecting the interaction between generational experience and issue
salience.

\subsection{Heterogeneity by region and employment}

\begin{figure}[t] % t: ページ上部, h: その場, b: 下部 など
  \centering
  \includegraphics[width=1\textwidth]{figure_ICT_marginal_PRblock_by_territory.pdf}
  \caption{Estimated share indifferent by territorial dispute}
  \label{fig:marginal_block}
\end{figure}

Figure~\ref{fig:marginal_block} examines how latent indifference varies across
proportional representation (PR) blocks. For Takeshima and the Senkaku Islands,
indifference is particularly high in metropolitan and urbanized regions such as Tokyo,
Kinki, and Kyushu. At the same time, residents in geographically proximate areas
(e.g.\ Tohoku for the Senkaku Islands; Hokkaido and Tohoku for the Northern
Territories) tend to be less indifferent, and in several cases the estimated marginal
effects are small and statistically indistinguishable from zero. This pattern is
consistent with the idea that physical proximity to the disputed areas increases the
perceived relevance of territorial issues, thereby lowering indifference.

The Northern Territories provide a more nuanced picture. While the national average
indifference is relatively low, Hokkaido---the region geographically closest to the
disputed islands---shows substantially higher indifference than most other blocks.
One plausible interpretation is that residents in Hokkaido may place more weight on
pragmatic concerns in their everyday lives, such as local economic conditions, than on
symbolic sovereignty claims, even under heightened tensions with Russia.

\begin{figure}[t] % t: ページ上部, h: その場, b: 下部 など
  \centering
  \includegraphics[width=0.9\textwidth]{figure_ICT_marginal_employment_by_territory.pdf}
  \caption{Estimated share indifferent by territorial dispute}
  \label{fig:marginal_employment}
\end{figure}

Figure~\ref{fig:marginal_employment} further disaggregates the AMEs by employment
status. For Takeshima and the Senkaku Islands, non-regular workers exhibit the
highest levels of latent indifference, followed by those who are not currently working
(students, homemakers, or the unemployed). Among the self-employed, indifference is
particularly pronounced for the Northern Territories. In contrast, regular employees
display lower and often statistically insignificant levels of indifference across all
three disputes.

These patterns suggest that precarious or non-standard labor market positions are
associated with weaker attachment to territorial claims. One interpretation is that
material insecurity and everyday economic concerns crowd out attention to foreign
policy issues, including disputes that are highly salient in elite political discourse.

\subsection{Robustness and diagnostic checks}

Because list experiments are sensitive to design failures, I conduct several standard
diagnostic checks recommended in the literature.

First, I examine potential floor and ceiling effects. For each dispute and experimental
group, I compute the share of respondents who select zero items and the share who
select the maximum possible number of items on the list. Appendix Figure~\ref{fig:appendix_floor_ceiling} visualizes the distribution of item counts by territory and experimental group. Across all three disputes,
between 29--34\% of respondents in the control and treatment groups report zero
items, while less than 4\% choose the maximum number of items. The absence of
substantial mass at the extremes suggests that neither floor nor ceiling effects are
severe enough to threaten the validity of the ICT estimates.

Second, I check whether the non-sensitive control items behave similarly across
experimental conditions. Focusing on respondents in the control condition, the mean
number of agreed non-sensitive items is highly similar across the Takeshima,
Senkaku, and Northern Territories samples, and their confidence intervals overlap
substantially. Appendix Figure~\ref{fig:appendix_control_means} summarizes the mean number of control items by territory.
This indicates that the lists share a comparable baseline level of
sensitivity and that differences across disputes are not driven by idiosyncrasies in the
control items.

Third, I implement a placebo test. I construct a dataset that combines the control
group with each treatment group and regress the count of non-sensitive items on a
pseudo-treatment indicator and the full set of covariates. Because the sensitive item is
not included in these counts, the treatment indicator should have no effect if the
randomization and the design are valid. Consistent with this expectation, the
treatment coefficient is effectively unidentified (and statistically null) once the model
drops the collinear indicator, and the covariate patterns are stable across placebo
specifications. The specification and placeholder results are reported in Appendix Table~\ref{tab:placebo_reg}.
This strengthens confidence that the positive AMEs reported above are
capturing genuine responses to the sensitive item, rather than artefacts of the survey
design.

Taken together, these diagnostic checks suggest that the list experiment performs as
intended: extreme response behaviour is limited, the control items are well behaved,
and the treatment assignment does not spuriously shift responses to the
non-sensitive items.

\section{Discussion}

The list experiment reveals a sizable, and previously hidden, reservoir of indifference
toward Japan's territorial disputes. Roughly one in five citizens are estimated to be
indifferent to the sovereignty of Takeshima, about one in nine to the Senkaku
Islands, and one in twenty to the Northern Territories. These estimates are
substantially lower than the levels of ``concern'' reported in conventional Cabinet
Office surveys, in which strong majorities claim to care about each dispute when
asked directly. The gap between direct and indirect measures is consistent with the
presence of social desirability bias: citizens feel pressure to express patriotic concern
even when they lack strong personal views.

At the same time, indifference is systematically structured. It is not simply random
noise or a residual category of ``don't know'' responses. The ICT regression and
marginal effect analyses show that indifference varies by territory, region, age, and
employment status in ways that are broadly consistent with existing theories of
territorial politics and public opinion.

For Takeshima, hidden indifference is most pronounced. This is striking given the
high symbolic salience of the Liancourt Rocks in Japan--Korea relations and the
prominent role of the dispute in nationalist rhetoric. One plausible interpretation is
that the intensity of public concern has been overstated by direct surveys that are
highly susceptible to social desirability bias. Although elites in both countries have
invested heavily in nationalist framing, a considerable share of citizens appear
willing---at least privately---to live with ambiguity over sovereignty.

The Senkaku Islands occupy an intermediate position. Indifference is lower than for
Takeshima but still far from negligible. The dispute is embedded in a broader
geopolitical rivalry with China and is frequently framed in security terms, which may
make it harder for citizens to admit indifference even under the protection of an
indirect question. Yet the presence of a non-trivial indifferent group suggests that
domestic constraints on compromise may be less rigid than aggregate survey
responses imply.

The Northern Territories present a different pattern. Nationally, the estimated share
of indifferent citizens is relatively small, which is consistent with the heightened
salience of Russia as an adversary since the 2022 invasion of Ukraine. However,
residents of Hokkaido, geographically closest to the islands and most
directly affected by cross-border interactions, display substantially higher latent
indifference. This divergence between national and local attitudes challenges the
assumption that proximity to disputed territories always produces hardline views. In
the Northern Territories case, everyday concerns and pragmatic local interests may
temper symbolic attachment to sovereignty.

Employment status further conditions territorial attitudes. Non-regular workers and
those outside the labour force are more likely to be indifferent to Takeshima and the
Senkaku Islands, while self-employed respondents are relatively indifferent to the
Northern Territories. These patterns resonate with research showing that economic
insecurity and precarious work can depress political engagement and shift attention
away from foreign policy and nationalist issues. In other words, indifference is
socio-economically stratified: citizens facing unstable livelihoods are less invested in
symbolic disputes over distant islands.

These findings have implications for debates on the domestic politics of foreign
policy. First, they nuance the standard picture in which territorial disputes are treated
as indivisible issues backed by uniformly hawkish publics. Even on highly salient
disputes, a sizeable minority of citizens are quietly indifferent. Second, they add a
micro-level perspective to the literature on how elites strategically construct
territorial issues. While past work has shown that governments can mobilize public
opinion around territorial symbols, the present results suggest that such efforts may
coexist with a reservoir of latent indifference that is invisible in direct surveys.

Finally, the results speak to the potential for diplomatic flexibility. If policymakers
recognize that public opinion is more nuanced than headline survey figures suggest,
there may be more room to explore creative arrangements that de-emphasize formal
sovereignty while addressing local economic and security concerns. At the same
time, the concentration of indifference among precarious groups raises normative
questions about whose preferences are most likely to be heard in the foreign policy
process. Future research should investigate the stability of these indifferent attitudes
over time and their responsiveness to elite framing, as well as triangulate list
experiments with other methods such as qualitative interviews and panel surveys.

\section{Conclusion}
Conclusion goes here.

\section*{Acknowledgments}

I thank JX Press Corporation for providing access to the survey data used in this study.
The views expressed here are my own and do not represent those of the company.

\section*{Data Availability}
The survey data used in this paper were provided by JX Press Corporation and
cannot be publicly shared due to contractual and confidentiality restrictions.
Replication code using simulated example data is available upon request.

\section*{Conflict of Interest}
The author has an outsourcing contract with JX Press Corporation,
which provided access to the data used in this study.
The company had no role in the design, analysis, or conclusions of this research.


\bibliography{references}

\clearpage
\appendix
\section*{Appendix}
\addcontentsline{toc}{section}{Appendix}

% Appendix 図表を A1, A2, ... 表記にする場合(推奨)
\renewcommand{\thetable}{A\arabic{table}}
\renewcommand{\thefigure}{A\arabic{figure}}
\setcounter{table}{0}
\setcounter{figure}{0}

\section{Full ICT Regression Results}
\begin{table}[htbp]
\centering
\caption{ICT Regression Results for Three Territorial Disputes (Key Coefficients)}
\label{tab:ict_regression}
\begin{threeparttable}
\scriptsize % small → scriptsize に変更
\setlength{\tabcolsep}{2pt} % 列の左右の余白を狭くする
\renewcommand{\arraystretch}{1.1} % 行間を詰める
\begin{tabular}{lccc}
\toprule
& Takeshima & Senkaku & Northern Territories \\
\midrule
Treatment (List) & \num{-0.120} & \num{0.235} & \num{0.348} \\
& (\num{0.296}) & (\num{0.316}) & (\num{0.305}) \\
Treatment × Male & \num{-0.157} & \num{-0.175}+ & \num{-0.152} \\
& (\num{0.096}) & (\num{0.093}) & (\num{0.093}) \\
Treatment × Other gender & \num{-0.565} & \num{-0.309} & \num{0.005} \\
& (\num{0.387}) & (\num{0.397}) & (\num{0.396}) \\
Treatment × Age: 30s & \num{-0.010} & \num{0.057} & \num{-0.634}* \\
& (\num{0.263}) & (\num{0.261}) & (\num{0.265}) \\
Treatment × Age: 40s & \num{0.165} & \num{0.092} & \num{-0.528}* \\
& (\num{0.252}) & (\num{0.254}) & (\num{0.257}) \\
Treatment × Age: 50s & \num{-0.152} & \num{-0.205} & \num{-0.656}* \\
& (\num{0.251}) & (\num{0.249}) & (\num{0.256}) \\
Treatment × Age: 60s & \num{-0.075} & \num{0.097} & \num{-0.323} \\
& (\num{0.256}) & (\num{0.254}) & (\num{0.260}) \\
Treatment × Age: 70+ & \num{-0.074} & \num{-0.124} & \num{-0.408} \\
& (\num{0.249}) & (\num{0.248}) & (\num{0.255}) \\
Treatment × Hokkaido & \num{0.498}+ & \num{0.146} & \num{0.545}+ \\
& (\num{0.286}) & (\num{0.292}) & (\num{0.285}) \\
Treatment × Tohoku & \num{0.435} & \num{0.011} & \num{0.159} \\
& (\num{0.278}) & (\num{0.299}) & (\num{0.280}) \\
Treatment × Kita-Kanto & \num{0.339} & \num{-0.073} & \num{0.202} \\
& (\num{0.244}) & (\num{0.260}) & (\num{0.239}) \\
Treatment × Minami-Kanto & \num{0.292} & \num{-0.018} & \num{0.251} \\
& (\num{0.232}) & (\num{0.250}) & (\num{0.227}) \\
Treatment × Tokyo & \num{0.710}** & \num{0.133} & \num{0.356} \\
& (\num{0.238}) & (\num{0.258}) & (\num{0.234}) \\
Treatment × Hokuriku-Shinetsu & \num{0.558}* & \num{0.263} & \num{0.449} \\
& (\num{0.279}) & (\num{0.299}) & (\num{0.275}) \\
Treatment × Tokai & \num{0.263} & \num{-0.049} & \num{0.101} \\
& (\num{0.241}) & (\num{0.258}) & (\num{0.235}) \\
Treatment × Kinki & \num{0.579}* & \num{-0.044} & \num{0.383}+ \\
& (\num{0.227}) & (\num{0.245}) & (\num{0.219}) \\
Treatment × Shikoku & \num{0.139} & \num{-0.339} & \num{-0.055} \\
& (\num{0.356}) & (\num{0.383}) & (\num{0.339}) \\
Treatment × Kyushu & \num{0.556}* & \num{-0.054} & \num{0.285} \\
& (\num{0.255}) & (\num{0.275}) & (\num{0.253}) \\
\midrule
Num.Obs. & \num{2253} & \num{2275} & \num{2266} \\
R2       & \num{0.038} & \num{0.026} & \num{0.033} \\
F        & \num{2.469} & \num{1.690} & \num{2.154} \\
RMSE     & \num{1.10}  & \num{1.09}  & \num{1.08} \\
\bottomrule
\end{tabular}
\begin{tablenotes}[flushleft]
\footnotesize
\item + p $<$ 0.1, * p $<$ 0.05, ** p $<$ 0.01, *** p $<$ 0.001
\item OLS models of list experiment item counts. Coefficients on Treatment and its interactions capture latent indifference; coefficients without Treatment are omitted for brevity.
\end{tablenotes}
\end{threeparttable}
\end{table}

\FloatBarrier

\section{Robustness Checks}

\subsection{Floor and Ceiling Effects}

\begin{center}
  \includegraphics[width=0.75\textwidth]{figure_ICT_hist_Y_by_territory_group.pdf}
  \captionof{figure}{Distribution of item counts by territory and experimental group}
  \label{fig:appendix_floor_ceiling}
\end{center}

\subsection{Baseline Comparability of Control Items Across Territories}

\begin{center}
  \includegraphics[width=0.75\textwidth]{figure_controlgroup_mean_by_territory.pdf}
  \captionof{figure}{Mean number of control items by territory}
  \label{fig:appendix_control_means}
\end{center}

\subsection{Placebo Regression (pseudo-treatment)}

As a placebo test, I regress the number of non-sensitive control items on the pseudo-treatment indicator
and covariates using the control list only.

\begin{table}[htbp]
\centering
\caption{Placebo Regression Using Control-Item Counts}
\label{tab:placebo_reg}
\begin{threeparttable}
\scriptsize % small → scriptsize に変更
\renewcommand{\arraystretch}{1.00} % 行間を詰める
\begin{tabular*}{\textwidth}{@{\extracolsep{\fill}}lc}
\toprule
& Placebo \\
\midrule
Intercept & \num{1.224}*** \\
& (\num{0.205}) \\
Male & \num{0.082} \\
& (\num{0.072}) \\
Other gender & \num{-0.075} \\
& (\num{0.267}) \\
Age: 30s & \num{0.201} \\
& (\num{0.176}) \\
Age: 40s & \num{0.075} \\
& (\num{0.170}) \\
Age: 50s & \num{0.190} \\
& (\num{0.168}) \\
Age: 60s & \num{0.249} \\
& (\num{0.173}) \\
Age: 70+ & \num{0.373}* \\
& (\num{0.174}) \\
Hokkaido & \num{-0.080} \\
& (\num{0.185}) \\
Tohoku & \num{-0.185} \\
& (\num{0.182}) \\
Kita-Kanto & \num{0.058} \\
& (\num{0.154}) \\
Minami-Kanto & \num{0.136} \\
& (\num{0.147}) \\
Tokyo & \num{-0.045} \\
& (\num{0.153}) \\
Hokuriku-Shinetsu & \num{-0.200} \\
& (\num{0.179}) \\
Tokai & \num{0.055} \\
& (\num{0.155}) \\
Kinki & \num{-0.014} \\
& (\num{0.142}) \\
Shikoku & \num{0.272} \\
& (\num{0.246}) \\
Kyushu & \num{-0.080} \\
& (\num{0.162}) \\
Non-regular & \num{-0.515}*** \\
& (\num{0.103}) \\
Self-employed & \num{-0.244}+ \\
& (\num{0.133}) \\
Not working & \num{-0.509}*** \\
& (\num{0.091}) \\
Other employment & \num{-0.327}** \\
& (\num{0.117}) \\
\midrule
Num.Obs. & \num{1147} \\
R2 & \num{0.060} \\
\bottomrule
\end{tabular*}
\begin{tablenotes}[flushleft]
\footnotesize
\item + p $<$ 0.1, * p $<$ 0.05, ** p $<$ 0.01, *** p $<$ 0.001.
\item Dependent variable: number of non-sensitive control items (0--4).
\item The pseudo-treatment indicator is perfectly collinear with the intercept and therefore omitted.
\item The same control group is used for all disputes; hence a single-column placebo test is sufficient.
\end{tablenotes}
\end{threeparttable}
\end{table}

\section{Regional Coding: Prefectures and PR Blocks}

Table~\ref{tab:pr_blocks} shows how Japan's 47 prefectures are grouped into
the eleven proportional representation (PR) electoral districts used in the analysis.

\begin{table}[htbp]
  \centering
  \caption{Mapping of prefectures to PR electoral districts}
  \label{tab:pr_blocks}
  \small
  \begin{tabular}{ll}
    \toprule
    PR block            & Prefectures \\
    \midrule
    Hokkaido            & Hokkaido \\
    Tohoku              & Aomori, Iwate, Miyagi, Akita, Yamagata, Fukushima \\
    Kita-Kanto          & Ibaraki, Tochigi, Gunma, Saitama \\
    Minami-Kanto        & Chiba, Kanagawa, Yamanashi \\
    Tokyo               & Tokyo \\
    Hokuriku-Shinetsu   & Niigata, Toyama, Ishikawa, Fukui, Nagano \\
    Tokai               & Gifu, Shizuoka, Aichi, Mie \\
    Kinki               & Shiga, Kyoto, Osaka, Hyogo, Nara, Wakayama \\
    Chugoku             & Tottori, Shimane, Okayama, Hiroshima, Yamaguchi \\
    Shikoku             & Tokushima, Kagawa, Ehime, Kochi \\
    Kyushu              & Fukuoka, Saga, Nagasaki, Kumamoto, Oita, Miyazaki, Kagoshima, Okinawa \\
    \bottomrule
  \end{tabular}
\end{table}



\end{document}